% \documentclass[12pt]{article}

% \usepackage{pgf-pie}  % For pie charts
% \usepackage{currfile} % Required for getting the current file name
% \usepackage{tikz}     % Required for drawing graphics
% \usepackage{pgfplots}
% \usepackage{pgfplotstable}
% \pgfplotsset{compat=newest}
% \usepackage{underscore}

% \makeatletter
% \newcommand\resetstackedplots{
% \makeatletter
% \pgfplots@stacked@isfirstplottrue
% \makeatother
% }

% \begin{document}

\pgfplotstableread[col sep=comma]{\currfiledir/M2.csv}\dataA
\pgfplotstableread[col sep=comma]{\currfiledir/M3.csv}\dataB
\pgfplotstableread[col sep=comma]{\currfiledir/M4.csv}\dataC
\pgfplotstableread[col sep=comma]{\currfiledir/M5.csv}\dataD


\begin{figure}

  \begin{tikzpicture}[scale=0.7]
    \begin{axis}[
      width=\textwidth, height=0.6172\textwidth,
      xlabel={ Number of tasks }, ylabel={ execution time (s) },
      xticklabels from table={\dataA}{resources.tasks},
      xtick=data, xtick align=outside,
      ymajorgrids=true, yminorgrids=true,
      bar width=7pt,
      ybar stacked,
      legend style={at={(1.01,1)},anchor=north west}
    ]

      \addplot+[ybar, bar width=0.2,point meta=y,draw=black,fill=red ] table [x expr=\coordindex+0.25*1, y=Export] {\dataA} ;
      \addlegendentry{ LoadMesh }
      \addplot+[ybar, bar width=0.2,point meta=y,draw=black,fill=green ] table [x expr=\coordindex+0.25*1, y=CreateFunctionSpace] {\dataA} ;
      \addlegendentry{ CreateFunctionSpace }
      \addplot+[ybar, bar width=0.2,point meta=y,draw=black,fill=blue ] table [x expr=\coordindex+0.25*1, y=LoadMesh] {\dataA} ;
      \addlegendentry{ Export }
      \resetstackedplots
      \addplot+[ybar, bar width=0.2,point meta=y,draw=black,fill=red, forget plot ] table [x expr=\coordindex+0.25*2, y=Export] {\dataB} ;
      \addplot+[ybar, bar width=0.2,point meta=y,draw=black,fill=green, forget plot ] table [x expr=\coordindex+0.25*2, y=CreateFunctionSpace] {\dataB} ;
      \addplot+[ybar, bar width=0.2,point meta=y,draw=black,fill=blue, forget plot ] table [x expr=\coordindex+0.25*2, y=LoadMesh] {\dataB} ;
      \resetstackedplots
      \addplot+[ybar, bar width=0.2,point meta=y,draw=black,fill=red, forget plot ] table [x expr=\coordindex+0.25*3, y=Export] {\dataC} ;
      \addplot+[ybar, bar width=0.2,point meta=y,draw=black,fill=green, forget plot ] table [x expr=\coordindex+0.25*3, y=CreateFunctionSpace] {\dataC} ;
      \addplot+[ybar, bar width=0.2,point meta=y,draw=black,fill=blue, forget plot ] table [x expr=\coordindex+0.25*3, y=LoadMesh] {\dataC} ;
      \resetstackedplots
      \addplot+[ybar, bar width=0.2,point meta=y,draw=black,fill=red, forget plot ] table [x expr=\coordindex+0.25*4, y=Export] {\dataD} ;
      \addplot+[ybar, bar width=0.2,point meta=y,draw=black,fill=green, forget plot ] table [x expr=\coordindex+0.25*4, y=CreateFunctionSpace] {\dataD} ;
      \addplot+[ybar, bar width=0.2,point meta=y,draw=black,fill=blue, forget plot ] table [x expr=\coordindex+0.25*4, y=LoadMesh] {\dataD} ;

    \end{axis}
  \end{tikzpicture}

  \caption{ General Performance (P1) }
\end{figure}

% \end{document}